%----------------------------------------------------------------------------------------
%  ACKNOWLEDGEMENTS
%----------------------------------------------------------------------------------------

\renewcommand{\epigraphflush}{flushleft}
\renewcommand{\sourceflush}{flushleft}
\setlength{\epigraphwidth}{0.8\textwidth}
\setlength{\epigraphrule}{0pt}

\begin{acknowledgements}
  \addchaptertocentry{\acknowledgementname} % Add the acknowledgements to the table of contents
  
 \epigraph{
Jučer, još se srećom osmjehiv'o dan,\\
danas izgubih biser u travi.\\
I zalud mi ga je tražiti, znam.\\
\vspace{0.5em}
Ali ja isto tako znam\\
da će sutra doći netko drugi\\
\vspace{0.5em}
i naći moj biser,\\
dosnivati izgubljen san.}{\hspace{1em}\textit{---Vesna Krmpotić}}
  
  
My view of the scientific process is that of stochastic gradient descent --- we are all clumsily stumbling down the gradient of ignorance. I hope that this thesis presents a small incremental step in the right direction. If not, then at least I hope you find the local minima discovered here useful in your own stumbling. 

Of course, like all things, this thesis only arose as a result of sufficient conditions, including people. I can only take partial credit, so here I would like to acknowledge a small subset of the people who made this thesis possible.
  
First, I would like to thank you supervisor in Osijek prof.\@ Irena Galić for always providing support, teaching me how to navigate the academic environment, and often saving me from my own bad decisions. I would also like to thank my supervisor in Ghent prof.\@ Aleksandra Pižurica for lending her great expertise. I thank my other supervisor in Ghent, dr.\@ Danilo Babin, for always putting my work in the context of medical imaging and expanding my professional network.
  
Thank you to all of the members of the examination board for taking the time out of your busy schedules to provide valuable feedback and to participate in the evaluation process. Thank you to the Croatian Science Foundation for providing the funding for my Ph.D.\@ studies.
  
Alongside my official mentors, over the years I have picked up a few unofficial ones as well. My office mate and colleague dr.\@ Marija Habijan was always there to patiently answer my endless questions and provide support. My dear friend Matej Džijan, whether he liked it or not, was on the receiving end of many half-thought-out bad ideas that he promptly challenged and improved. Thank you both!

I would also like to thank the friends, coauthors, and colleagues who joined me on my Ph.D.\@ journey, including prof.\@ H. Leventić and prof.\@ K. Romić from my research group at FERIT, as well as S. Lazendić, N. Vercheval, N. Žižakić, N. Nadisic, Y. Arhant, Y. Qiu, and X. Li from the Group for Artificial Intelligence and Sparse Modelling in Ghent.

Thank you to my parents and family for always being there for me. Finally, I would like to thank my wife Dora for making my life infinitely more joyous. 

This thesis is dedicated to my favorite person in the world, my son Javor.

\setlength{\epigraphwidth}{0.3\textwidth}
\renewcommand{\epigraphflush}{flushright}
 \epigraph{
Marin Benčević\\
January, 2024.
}{}
\end{acknowledgements}
