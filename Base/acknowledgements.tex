%----------------------------------------------------------------------------------------
%  ACKNOWLEDGEMENTS
%----------------------------------------------------------------------------------------

\renewcommand{\epigraphflush}{flushleft}
\renewcommand{\sourceflush}{flushleft}
\setlength{\epigraphwidth}{0.8\textwidth}
\setlength{\epigraphrule}{0pt}

\begin{acknowledgements}
  \addchaptertocentry{\acknowledgementname} % Add the acknowledgements to the table of contents
  
 \epigraph{ 
Ne boj se, nisi sam! Ima i drugih nego ti\\
koji nepoznati od tebe žive tvojim životom.\\
I ono sve što ti bje, ču i što sni\\
gori u njima istim žarom, ljepotom i čistotom.\\[0.5em]
Ne gordi se! Tvoje misli nisu samo tvoje! One u drugima žive.\\
Mi smo svi prešli iste putove u mraku,\\
mi smo svi jednako lutali u znaku\\
traženja, i svima jednako se dive.
}{\hspace{1em}\textit{---Tin Ujević}}
  
  
My view of the scientific process is that of stochastic gradient descent --- we are all clumsily stumbling down the gradient of ignorance. I hope that this thesis presents a small incremental step in the right direction. If not, then at least I hope you find the local minima discovered here useful in your own stumbling. 

Of course, like all things, this thesis only arose as a result of sufficient conditions, including people. I can only take partial credit, so here I would like to acknowledge a small subset of the people who made this thesis possible.
  
First, I would like to thank you supervisor in Osijek prof.\@ Irena Galić for always providing support, teaching me how to navigate the academic environment, and often saving me from my own bad decisions. I would also like to thank my supervisor in Ghent prof.\@ Aleksandra Pižurica for lending her great expertise. I thank my other supervisor in Ghent, dr.\@ Danilo Babin, for always putting my work in the context of medical practice and expanding my professional network.
  
Thank you to all of the members of the examination board for taking the time out of your busy schedules to provide valuable feedback and to participate in the evaluation process.
  
Alongside my official mentors, over the years I have picked up a few unofficial ones as well. My office mate and colleague Marija Habijan was always there to patiently answer my endless questions and provide support. My dear friend Matej Džijan, whether he liked it or not, was on the receiving end of many half-thought-out bad ideas that he promptly challenged and improved. Thank you both!

I would also like to thank the friends, coauthors, and colleagues who joined me on my Ph.D.\@ journey, including H. Leventić, K. Romić and R. Šojo from the Research Group for Computer Science and Human-Computer Interaction in Osijek, as well as S. Lazendić, N. Vercheval, N. Žižakić, N. Nadisic, Y. Arhant, Y. Qiu, and X. Li from the Group for Artificial Intelligence and Sparse Modelling in Ghent. 

I would like to acknowledge the support of FERIT for employing me during my Ph.D. and all of its staff for their support. This work has also been supported in part by the Croatian Science Foundation under the project UIP-2017-05-4968.

Thank you to my parents and family for always being there for me. Finally, I would like to thank my wife Dora for making my life infinitely more joyous. 

This thesis is dedicated to my favorite person in the world, my son Javor.

\setlength{\epigraphwidth}{0.3\textwidth}
\renewcommand{\epigraphflush}{flushright}
 \epigraph{
Marin Benčević\\
January, 2024.
}{}
\end{acknowledgements}
