%----------------------------------------------------------------------------------------
%  ABSTRACT PAGE
%----------------------------------------------------------------------------------------
\chapter{Samenvatting}

Medische beeldsegmentatie speelt een centrale rol in verschillende medische toepassingen, van chirurgische planning tot diagnose en onderzoek. Over het algemeen steunt medische beeldsegmentatie sterk op neurale netwerken om nauwkeurige segmentatie te bereiken bij verschillende beeldmodaliteiten zoals CT, MRI, microscopie, dermatoscopie en andere. De effectiviteit van neurale netwerken wordt echter aanzienlijk beïnvloed door de beschikbaarheid en kwaliteit van trainingsgegevens, die vaak moeilijk te verkrijgen zijn vanwege de hoge tijd en financiële kosten van beeldacquisitie, de invasiviteit van sommige beeldvormingsprocedures, grote bestandsgroottes en wettelijke uitdagingen. De arbeidsintensieve en deskundige aard van het annoteren van medische beelden voor segmentatie maakt deze uitdagingen nog groter, waardoor het bijzonder moeilijk is om grote datasets van hoge kwaliteit samen te stellen voor het trainen van segmentatiemodellen voor medische beelden.

Statistische leertheoretische principes geven aan dat complexere medische beeldsegmentatietaken neurale netwerken met een groter aantal parameters vereisen voor effectieve segmentatie. Deze vereiste voor meer parameters vereist op zijn beurt grotere steekproeven om overpassen te voorkomen. In de medische beeldvorming is er echter een gebrek aan datasets met een grote steekproefomvang. Daarom is er een grote behoefte aan gegevensefficiënte segmentatiemethoden die betrouwbare resultaten kunnen leveren met beperkte trainingssamples. 

In dit proefschrift presenteren we een overzicht van segmentatiemethoden voor medische afbeeldingen en bestaande strategieën om hun gegevensefficiëntie te verbeteren. Daarnaast stellen we verschillende nieuwe methoden voor die de segmentatietaak vereenvoudigen, waardoor convolutionele neurale netwerken nauwkeurige segmentatie kunnen uitvoeren met minder parameters en, bij uitbreiding, kleinere steekproeven. Onze methoden zijn gericht op het transformeren van de segmentatiegrens in een representatie die met minder parameters gemodelleerd kan worden. We doen dit door domeinkennis en traditionele beeldverwerkingstechnieken te gebruiken om gunstige beeldtransformaties te identificeren. De parameters van de beeldtransformaties worden dynamisch geselecteerd voor elk beeld met behulp van neurale netwerken, waardoor de segmentatietaak wordt opgesplitst in twee beter hanteerbare fasen: een eerste ruwe lokalisatie van het doelobject gevolgd door de segmentatie van een vereenvoudigde weergave van het beeld.

Om de uitdaging van gegevensefficiëntie in medische beeldsegmentatie aan te gaan, introduceert dit proefschrift de volgende originele wetenschappelijke bijdragen:

\begin{enumerate}
	\item \textbf{Een nieuwe biomedische beeldsegmentatiemethode gebaseerd op polaire transformatie voorbewerking met een geleerd middelpunt.} We introduceren een nieuwe voorbewerkingstechniek voor medische beelden, met name beelden met elliptisch gevormde objecten. Door een polaire transformatie op de afbeelding toe te passen, vereenvoudigen we cirkelvormige beslissingsgrenzen in lineaire, waardoor segmentatie eenvoudiger wordt. Een belangrijke bijdrage is de ontwikkeling van een neuraal netwerk om de optimale oorsprong voor de polaire transformatie te identificeren, waardoor de segmentatieprestaties verbeteren en het gebruik van minder complexe netwerken mogelijk wordt.
	\item \textbf{Een verbeterde methode voor het verkleinen van de invoerafbeelding in neurale netwerken met behulp van saillante beeldgewassen.} Voortbouwend op de inzichten van modelgedreven polaire transformaties, stellen we een modelgedreven bijsnijdtechniek voor om de invoerafmetingen van neurale netwerken te minimaliseren zonder fijne details op te offeren. Door het doelobject te lokaliseren in een gedownsampled beeld en geïdentificeerde interessegebieden te extraheren uit een beeld met hoge resolutie, behouden we nauwkeurige segmentatie met kleinere netwerkingangsgroottes. Aangezien de complexiteit van het model toeneemt met de afbeeldingsgrootte, verbetert deze reductie van de inputgrootte de gegevensefficiëntie.
	\item \textbf{Een nieuwe neurale netwerkarchitectuur voor beeldsegmentatie met hoge resolutie die objectdetectie in beelden met lage resolutie en segmentatie in beelden met hoge resolutie combineert.} We breiden onze voorbewerkingsmethoden uit om een end-to-end trainbaar netwerk te creëren dat objectlokalisatie in lage resolutie combineert met beeldsegmentatie in hoge resolutie. We maken netwerkconvergentie op twee belangrijke manieren mogelijk. Ten eerste gebruiken we een gedeelde architectuur tussen de lokalisatie- en segmentatiesubnetwerken, wat transferleren mogelijk maakt. Ten tweede zorgen we ervoor dat de gradiënt door het netwerk stroomt door informatie van het ene subnetwerk door te geven aan het andere. We laten zien dat het trainen van dit end-to-end netwerk de robuustheid van zowel de lokalisatie- als de segmentatiefase verhoogt.
	\item \textbf{Een nieuwe methode om diepte-informatie in te sluiten in tweedimensionale convolutionele neurale netwerkingangsdata.} Met het oog op de beperkingen van de gegevensefficiëntie van 3D-netwerken voor volumetrische gegevens zoals CT-scans, ontwikkelen we een methode om diepte-informatie in te sluiten in 2D-doorsneden door een genormaliseerd z-coördinaatkanaal toe te voegen aan elke doorsnede. We laten zien dat dit effectieve segmentatie van CT-beelden met 2D netwerken op basis van plakjes mogelijk maakt.
\end{enumerate}

De effectiviteit van deze methoden wordt gevalideerd voor verschillende modaliteiten voor medische beeldvorming, waaronder CT-scans, microscopie, dermatoscopie en colonoscopie, waarbij niet alleen een verbeterde gegevensefficiëntie wordt aangetoond, maar ook state-of-the-art segmentatieresultaten voor bepaalde taken. De geïntroduceerde technieken zijn veelzijdig, geschikt voor een breed spectrum van medische beeldvorming en kunnen dienen als algemene voorbewerkingsstappen voor elke op convolutionele neurale netwerken gebaseerde segmentatiearchitectuur.

Het in dit proefschrift gepresenteerde onderzoek heeft geresulteerd in de publicatie van vier onderzoekspapers in wetenschappelijke tijdschriften (allemaal als eerste auteur) en vier papers die op internationale wetenschappelijke conferenties werden gepresenteerd (waarvan drie als eerste auteur). 