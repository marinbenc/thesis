%----------------------------------------------------------------------------------------
%  ABSTRACT PAGE
%----------------------------------------------------------------------------------------
\chapter{Samenvatting}

Medische beeldsegmentatie speelt een centrale rol in verschillende medische toepassingen, van chirurgische planning tot diagnose en onderzoek. Over het algemeen steunt medische beeldsegmentatie sterk op neurale netwerken om nauwkeurige segmentatie te bereiken bij verschillende beeldmodaliteiten zoals CT, MRI, microscopie, dermatoscopie en andere. De effectiviteit van neurale netwerken wordt echter aanzienlijk beïnvloed door de beschikbaarheid en kwaliteit van training data, die vaak moeilijk te verkrijgen is vanwege de dure en langdurige acquisitie, de invasiviteit van sommige technieken, grote bestandsgroottes en wettelijke uitdagingen. De arbeidsintensieve en deskundige aard van het annoteren van medische beelden voor segmentatie maakt deze uitdagingen nog groter, waardoor het bijzonder moeilijk is om grote datasets van hoge kwaliteit samen te stellen voor het trainen van segmentatiemodellen voor medische beelden.

Statistische leertheoretische principes geven aan dat complexere medische beeldsegmentatie neurale netwerken met een groot aantal parameters vereist voor effectieve segmentatie. Dit vereist op zijn beurt grotere steekproeven om overfitting te voorkomen. In de medische beeldvorming is er echter een gebrek aan datasets van dergelijke omvang. Daarom is er een grote behoefte aan gegevensefficiënte segmentatiemethoden die betrouwbare resultaten kunnen leveren met beperkte training data. 

In dit proefschrift presenteren we een overzicht van segmentatiemethoden voor medische afbeeldingen en bestaande strategieën om hun gegevensefficiëntie te verbeteren. Daarnaast stellen we verschillende nieuwe methoden voor die de segmentatietaak vereenvoudigen, waardoor convolutionele neurale netwerken nauwkeurige segmentatie kunnen uitvoeren met minder parameters en, bij uitbreiding, kleinere steekproeven. Onze methoden zijn gericht op het transformeren van de segmentatiegrens in een representatie die met minder parameters gemodelleerd kan worden. We doen dit door domeinkennis en traditionele beeldverwerkingstechnieken te gebruiken om gunstige beeldtransformaties te identificeren. De parameters van de beeldtransformaties worden dynamisch geselecteerd voor elk beeld met behulp van neurale netwerken, waardoor de segmentatietaak wordt opgesplitst in twee gemakkelijkere hanteerbare fasen: een eerste ruwe lokalisatie van het doelobject gevolgd door de segmentatie van een vereenvoudigde weergave van het beeld.

Concreet gebruiken we twee neurale netwerken om segmentatietaken uit te voeren. Het eerste netwerk richt zich op het ruwweg lokaliseren van het doelobject, waarbij Gaussiaanse verdelingen gecentreerd op het object of eenvoudige segmentatiekaarten, als initiële richtlijnen worden gebruikt. We construeren een functie die, gegeven deze ruwe lokalisatie, parameters produceert voor de beeldtransformatie om de segmentatietaak te vereenvoudigen. Het besluitvormingsproces voor de functie en de beeldtransformaties is gebaseerd op zowel domeinkennis als empirische resultaten. Om bijvoorbeeld elliptische objecten efficiënter te segmenteren passen we de polaire transformatie toe, met de polaire oorsprong in het midden van het object. Het beeld ondergaat de beeldtransformatie volgens de resulterende parameters en wordt voorbereid voor het tweede neurale netwerk, dat dan gedetailleerde segmentatie uitvoert. Dit tweede netwerk wordt specifiek getraind op getransformeerde beelden met behulp van parameters verkregen uit segmentatiemaskers ter referentie.

Door het tweede netwerk te trainen op getransformeerde beelden, vereenvoudigen we de segmentatiegrens en kunnen we dus netwerken gebruiken met minder parameters. Netwerken met minder parameters vereisen minder gegevens om te trainen, en dus leidt onze aanpak van nature tot een toename in gegevensefficiëntie. Onze resultaten tonen een vergelijkbare of verbeterde nauwkeurigheid bij gebruik van minder data of labels in verschillende medische beeldvormingstoepassingen, zoals segmentatie van de lever en epicardiaal vetweefsel uit CT-scans, huidlaesies, poliepen uit colonoscopiebeelden en cellen uit microscopische beelden.

De twee neurale netwerken kunnen afzonderlijke netwerken zijn of ze kunnen na verbinding end-to-end worden getraind. Een opmerkelijk voordeel van deze aanpak is dat de ruwe lokalisatie- en fijne segmentatienetwerken dezelfde veelgebruikte architectuur voor medische beeldsegmentatie kunnen delen. Dit maakt transferleren tussen de twee netwerken eenvoudig, wat de trainingstijd verkort en de prestaties verbetert. Bovendien kunnen de netwerken worden voorgetraind met behulp van bestaande beschikbare datasets of netwerken voor medische beeldsegmentatie.

Om de uitdaging van gegevensefficiëntie in medische beeldsegmentatie aan te gaan, introduceert dit proefschrift de volgende originele wetenschappelijke bijdragen:

\begin{enumerate}
	\item \textbf{Een nieuwe biomedische beeldsegmentatiemethode gebaseerd op polaire transformatie voorbewerking met een geleerd middelpunt.} We introduceren een nieuwe voorbewerkingstechniek voor medische beelden met elliptisch gevormde objecten. Door een polaire transformatie op de afbeelding toe te passen, vereenvoudigen we cirkelvormige beslissingsgrenzen in lineaire, waardoor segmentatie eenvoudiger wordt. Een belangrijke bijdrage is de ontwikkeling van een neuraal netwerk om de optimale oorsprong voor de polaire transformatie te identificeren, waardoor de segmentatieprestaties verbeteren en het gebruik van minder complexe netwerken mogelijk wordt.
	\item \textbf{Een verbeterde methode voor het verkleinen van de invoerafbeelding in neurale netwerken met behulp van saillante bijgesneden afbeeldingen.} Voortbouwend op de inzichten van modelgedreven polaire transformaties, stellen we een modelgedreven techniek om bij te snijden voor om de invoerafmetingen van neurale netwerken te minimaliseren zonder fijne details op te offeren. Door het doelobject te lokaliseren in een gedownsampled beeld en geïdentificeerde interessegebieden te extraheren uit een beeld met hoge resolutie, behouden we nauwkeurige segmentatie met kleinere inputs voor het netwerk. Aangezien de complexiteit van het model toeneemt met de grootte van de afbeelding, verbetert deze reductie van de inputgrootte de gegevensefficiëntie.
	\item \textbf{Een nieuwe neurale netwerkarchitectuur voor beeldsegmentatie met hoge resolutie die objectdetectie in beelden met lage resolutie en segmentatie in beelden met hoge resolutie combineert.} We breiden onze voorbewerkingsmethoden uit om een end-to-end trainbaar netwerk te creëren dat objectlokalisatie in lage resolutie combineert met beeldsegmentatie in hoge resolutie. We maken netwerkconvergentie op twee belangrijke manieren mogelijk. Ten eerste gebruiken we een gedeelde architectuur tussen de lokalisatie- en segmentatie subnetwerken, wat transferleren mogelijk maakt. Ten tweede zorgen we ervoor dat de gradiënt door het netwerk stroomt door informatie van het ene subnetwerk door te geven aan het andere. We laten zien dat het trainen van dit end-to-end netwerk de robuustheid van zowel de lokalisatie- als de segmentatiefase verhoogt.
	\item \textbf{Een nieuwe methode om diepte-informatie te incorporeren in de data aan de ingang van tweedimensionale convolutionele neurale netwerken.} Met het oog op de beperkingen van de gegevensefficiëntie van 3D-netwerken voor volumetrische gegevens zoals CT-scans, ontwikkelen we een methode om diepte-informatie te incorporeren in 2D-doorsneden door een genormaliseerde z-coördinaat toe te voegen aan elke doorsnede. We demonstreren dat dit effectieve segmentatie van CT-beelden met 2D netwerken op basis van doorsneden mogelijk maakt.
\end{enumerate}

De effectiviteit van deze methoden wordt gevalideerd voor verschillende modaliteiten in medische beeldvorming, waaronder CT-scans, microscopie, dermatoscopie en colonoscopie, waarbij niet alleen een verbeterde gegevensefficiëntie wordt aangetoond, maar ook state-of-the-art segmentatieresultaten voor bepaalde taken. De geïntroduceerde technieken zijn veelzijdig, geschikt voor een breed spectrum van medische beeldvorming en kunnen dienen als algemene voorbewerkingsstappen voor elke op convolutionele neurale netwerken gebaseerde segmentatiearchitectuur.

Het onderzoek gepresenteerd in dit proefschrift resulteerde in de publicatie van vier onderzoekspapers in wetenschappelijke tijdschriften (allemaal als eerste auteur) en vijf papers die op internationale wetenschappelijke conferenties werden gepresenteerd (waarvan vier als eerste auteur). 